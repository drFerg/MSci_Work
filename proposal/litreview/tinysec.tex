TinySec is a fully functional symmetric security link layer component created for the wireless sensor network operating system, TinyOS. It was the first fully implemented solution for WSNs and was created to address the security worries of running a WSN and transmitting private sensor data in the clear. Unlike conventional security protocol implementations which can afford significant time and space overheads, such as 16-32 bytes for security per packet, WSN typically run on extremely constrained devices with packet sizes of just 30 bytes, making those implementations impossible to run.

To resolve this, TinySec took a balanced approach which made a compromise between the level of security, packet overhead and resource requirements. The end result proved that it's possible to secure a WSN efficiently entirely in software, without any additional hardware necessary.

\textbf{TODO: MORE TO WRITE ABOUT IMPLEMENTATION, SKIP-JACK and issues (key distribution problem, capture)}