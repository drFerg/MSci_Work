TinySec is a fully functional symmetric security link layer component created for the wireless sensor network operating system, TinyOS. It was the first fully implemented solution for WSNs and was created to address the security worries of running a WSN and transmitting private sensor data in the clear. Unlike conventional security protocol implementations which can afford significant time and space overheads, such as 16-32 bytes for security per packet, WSN typically run on extremely constrained devices with packet sizes of just 30 bytes, making those implementations impossible/extremely expensive to run.

To resolve this, TinySec took a balanced approach making a compromise between the level of security, packet overhead and resource requirements. The end result proved that it's possible to secure a WSN efficiently entirely in software, without the need for additional hardware. 

Communication between nodes, not just nodes-to-base-station, in WSNs is often quite important, allowing nodes to not only redirect other's traffic along routes but also consolidate duplicate packets from multiple nodes about the same event, saving the overall network from wasting power receiving and transmitting the extra packets; tinySec chose to engineer in security at the link layer, allowing these mechanisms to perform without alteration. The security goals of TinySec aimed to enable access control, whereby only authorised participants may participate in the network, with unauthorised messages easy to spot and reject; ensure message integrity, so that authorised messages can't be illegally altered by a man-in-the-middle without the receiver noticing; and ensure confidentiality, to ensure information is kept secret from unauthorised eavesdroppers. 

The TinySec implementation uses Cipher Block Chaining with an initialisation vector (IV), together these achieve semantic security, therefore ensuring that encrypting the same plain text twice returns a different cipher text each time. So that the receiving end knows how to begin decryption of the data, the IV must be sent in the clear along with the encrypted data. When using an IV, its length needs to be taken into consideration because repeats will occur when the number wraps, causing a security vulnerability. On unconstrained devices an IV is usually 8 or 16 bytes, however due to the packet size limitations of the wireless sensors used, a 8(2 byte counter) byte IV was chosen. In the IV, 6 bytes are made up of pre-existing fields to conserve space and ensure globally unique IVs in the network e.g. to nodes send the same data event and both happen to have the same counter value, but differ in source (src), so the IV is different, therefore preserving the security.

For ensuring authenticity and integrity of messages, TinySec uses Cipher Block Chaining Message Authentication Codes (CBC-MAC) of 4 bytes in length. Similar to a CRC, CBC-MAC runs over the data and produces a 4 byte MAC which is appended to the packet. If a message was to be altered, the attacker has a 1 in $2^{32}$ of blindly forging a valid MAC. In a WSN with a limited send rate of 19.2Kb/s it would take over 20 months to send enough packets to possibly succeed in forging a MAC. In the case of attack, a receive heuristic could be used to detect multiple failed MAC transmissions at a nearby node, triggering an alert to the rest of the network.

Whilst TinySec can secure a WSN against eavesdropping and forged messages, it has two significant drawbacks. Firstly, in regards to key distribution, encryption and authentication keys need to be loaded to the nodes prior to deployment. This can cause issues when the shared keys need to be changed, such as when they are compromised, as all nodes in the network will need the new key. This can be especially difficult post-deployment, simply due to the number of nodes and often embedded and/or difficult to reach locations. Secondly, if a node in the network is compromised, because the authentication key is a network wide shared key, the illegitimate node can pretend to be any other node in the network, making it difficult to protect, never mind counter against.