Similar to TinySec\cite{TinySec} and MiniSec\cite{MiniSec}, ContikiSec \cite{ContikiSec} is a symmetric cryptographic security network layer, however, it's built for the other significant WSN OS, Contiki, instead of TinyOS. The paper presents two main contributions; first, an extensive evaluation of several block-ciphers and modes-of-operation, comparing ROM/RAM sizes and timings; second, a modular symmetric encryption network layer for Contiki, offering modes for encryption, authentication or encryption + authentication.

After comparing several block ciphers (AES, Skipjack, RC5, Triple-DES, Twofish and XTEA), AES was chosen for use in ContikiSec due to its good trade-off between resource consumption and security. This is in contrast to both TinySec and MiniSec, which chose to use Skipjack. At the respective times of publication of TinySec and MiniSec, Skipjack was deemed sufficiently secure by NIST until 2008. Similar to MiniSec, ContikiSec also uses offset codebook as its mode of operation, combining both encryption and authentication into one pass.