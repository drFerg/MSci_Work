In the last decade, homes have become filled with a myriad of different computing devices, including PCs, laptops, tablets, phones, all with wireless connectivity; therefore requiring the user to install and manage a small-scale wireless network. Whilst networks have emerged in the home, similar to those found in a typical enterprise setting, the tools to manage them have remained largely untouched, designed for network engineers -- not home users. This leaves users to manage features such as access control, firewalls, port forwarding and many others, using primitive tools designed for experienced and qualified network managers. Because of this, networks are often badly managed or their features underutilised.

In an attempt to fundamentally change the home network, the Homework project\cite{Homework, HomeworkControl}, a re-invention of the protocols, models and architectures of the domestic setting led by ethnographic studies of use, has been created. By reinventing the home network from the ground up around the user, unlike typical face-lift attempts by manufacturers, its aim is to enable the typical home user to manage and be in control of their home network, using tools that induce a minimal overhead on the user. 

To do this, Homework utilises an information plane architecture that uses stream database concepts to process a raw event stream, generate derived events and allows monitoring applications to publish and subscribe to events in the stream \cite{InformationPlane}. Atop of this, a policy management engine, enables users to manage access and bandwidth control, using a user friendly and informative web interface, utilising the stream database to monitor the policies and display network statistics in a user-understandable way e.g. displaying a user's devices with progress bars displaying the percent bandwidth utilised.

As more networked devices enter the home, including the flood of devices soon to be brought by the Internet of Things, managing a home filled with multiple disconnected and heterogeneously filled networks will become an even more important and difficult task for the home user to perform. Thus, the task of intelligently merging these networks together and creation of an all in one network and information management system needs to be undertaken, ensuring that new attack vectors aren't exposed by the thoughtless integration of new devices. Using the Homework information plane architecture, this concept appears possible; even with the event streams filled with hundreds, if not thousands of events, generated by devices, including Things and typical computing devices, Homework appears ready to handle it\cite{DEBSChallenge}. Most importantly, the end result must abstract away the complexities, so that a typical home user can manage and maintain control over the network, utilising more complex and varying policies than before in order to manage the closed loop of control required for an IoT.