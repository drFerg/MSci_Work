As previously discussed, public key cryptography (PKC) using traditional algorithms, such as RSA, has had a limited deployment in WSN due to its high computational cost and implementation size. In cases where it has been used, implementations, such as TinyPK, only using a subset of the operations on the sensor node in order to reduce the overhead, but also reducing the effectiveness of it.

As an alternative to RSA and other typical algorithms, elliptic curve cryptography (ECC) shows promise for using PKC on WSNs, due to its lower computational overhead, enabling both public and private operations to be carried out on a sensor node, as well as it reduced key size and compact signatures, all the whilst maintaining the same security as RSA\cite{ECC} e.g. a 1024-bit RSA key size is equal in security to a 160-bit ECC key size. These benefits make it far for suitable for use on WSNs when compared to RSA, thus removing the need to offload work to other, more power, devices\cite{TinyPK}.

TinyECC \cite{TinyECC} is an implementation of ECC for the TinyOS WSN OS, featuring a full set of public and private key operations, unlike TinyPK. It also has various optimisation switches, allowing developers to balance implementation size against performance. TinyECC was tested on a variety of TinyOS-compatible constrained sensor platforms, including MICAz, TelosB, Tmote Sky and Imote2, extensively proving that it's feasible for a wide range of WSN platforms.

As mentioned in TinyPK, private key operations for signing blocks of data took in the order of tens of minutes\cite{TinyPK}, whereas, with the use of ECC, TinyECC achieves the same operation in 1.6s when all optimisations are used. Similarly, TinyECC also achieves encryption speeds of 3.3s/pkt and decryption speeds of 2.1s/pkt. Whilst these rates are several orders of magnitude slower than symmetric algorithms, they are only normally used for securely bootstrapping keys for symmetric cryptography, which would then be used instead.