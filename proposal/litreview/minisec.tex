MiniSec was created to tackle several problems apparent in the then current WSN security protocols, TinySec and Zigbee. The pre-cursor to MiniSec, TinySec, received much attention and use due to its power and resource efficient security implementation, but because of limited authentication and no replay prevention the overall security of it was deemed insufficient to protect a WSN. A commercial alternative, Zigbee, exhibits significantly higher security, but does so at the cost of high energy consumption. MiniSec was designed to find the middle-ground between the two, increasing security whilst remaining energy efficient. 


In contrast to TinySec, for its encryption mode of operation, MiniSec uses Offset Code Book. Unlike cipher block chaining (CBC) which requires two passes to provide both encryption and authentication, OCB provides both in only one pass over the data. This one pass also performs faster that CBC's two passes and only requires one key for operation, thus making it more appropriate for a constrained device in terms of power and storage. 

MiniSec also differs from TinySec by reducing the size of the IV counter sent in a packet, yet managing to increase the size of the IV so that it wraps less often. This is achieved by storing some state about the IV counter locally and only transmitting the last n bits of it to the receiving node. This also requires some logic on both sides to manage the counter in the event of packet loss larger than the range of values stored in the last n bits sent, i.e. when $>2^n$ are lost.

\textbf{TODO: Discuss improved security, improved E+AUTH with OCB, use of authentication (OCB) and replay prevention mechanism}