\documentclass[conference]{./sty/IEEEtran}
% Some very useful LaTeX packages include:
% (uncomment the ones you want to load)


% *** MISC UTILITY PACKAGES ***
%
%\usepackage{ifpdf}
% Heiko Oberdiek's ifpdf.sty is very useful if you need conditional
% compilation based on whether the output is pdf or dvi.
% usage:
% \ifpdf
%   % pdf code
% \else
%   % dvi code
% \fi
% The latest version of ifpdf.sty can be obtained from:
% http://www.ctan.org/tex-archive/macros/latex/contrib/oberdiek/
% Also, note that IEEEtran.cls V1.7 and later provides a builtin
% \ifCLASSINFOpdf conditional that works the same way.
% When switching from latex to pdflatex and vice-versa, the compiler may
% have to be run twice to clear warning/error messages.






% *** CITATION PACKAGES ***
%
\usepackage{cite}
% cite.sty was written by Donald Arseneau
% V1.6 and later of IEEEtran pre-defines the format of the cite.sty package
% \cite{} output to follow that of IEEE. Loading the cite package will
% result in citation numbers being automatically sorted and properly
% "compressed/ranged". e.g., [1], [9], [2], [7], [5], [6] without using
% cite.sty will become [1], [2], [5]--[7], [9] using cite.sty. cite.sty's
% \cite will automatically add leading space, if needed. Use cite.sty's
% noadjust option (cite.sty V3.8 and later) if you want to turn this off.
% cite.sty is already installed on most LaTeX systems. Be sure and use
% version 4.0 (2003-05-27) and later if using hyperref.sty. cite.sty does
% not currently provide for hyperlinked citations.
% The latest version can be obtained at:
% http://www.ctan.org/tex-archive/macros/latex/contrib/cite/
% The documentation is contained in the cite.sty file itself.






% *** GRAPHICS RELATED PACKAGES ***
%
\ifCLASSINFOpdf
  % \usepackage[pdftex]{graphicx}
  % declare the path(s) where your graphic files are
  % \graphicspath{{../pdf/}{../jpeg/}}
  % and their extensions so you won't have to specify these with
  % every instance of \includegraphics
  % \DeclareGraphicsExtensions{.pdf,.jpeg,.png}
\else
  % or other class option (dvipsone, dvipdf, if not using dvips). graphicx
  % will default to the driver specified in the system graphics.cfg if no
  % driver is specified.
  % \usepackage[dvips]{graphicx}
  % declare the path(s) where your graphic files are
  % \graphicspath{{../eps/}}
  % and their extensions so you won't have to specify these with
  % every instance of \includegraphics
  % \DeclareGraphicsExtensions{.eps}
\fi
% graphicx was written by David Carlisle and Sebastian Rahtz. It is
% required if you want graphics, photos, etc. graphicx.sty is already
% installed on most LaTeX systems. The latest version and documentation can
% be obtained at: 
% http://www.ctan.org/tex-archive/macros/latex/required/graphics/
% Another good source of documentation is "Using Imported Graphics in
% LaTeX2e" by Keith Reckdahl which can be found as epslatex.ps or
% epslatex.pdf at: http://www.ctan.org/tex-archive/info/
%
% latex, and pdflatex in dvi mode, support graphics in encapsulated
% postscript (.eps) format. pdflatex in pdf mode supports graphics
% in .pdf, .jpeg, .png and .mps (metapost) formats. Users should ensure
% that all non-photo figures use a vector format (.eps, .pdf, .mps) and
% not a bitmapped formats (.jpeg, .png). IEEE frowns on bitmapped formats
% which can result in "jaggedy"/blurry rendering of lines and letters as
% well as large increases in file sizes.
%
% You can find documentation about the pdfTeX application at:
% http://www.tug.org/applications/pdftex





% *** MATH PACKAGES ***
%
%\usepackage[cmex10]{amsmath}
% A popular package from the American Mathematical Society that provides
% many useful and powerful commands for dealing with mathematics. If using
% it, be sure to load this package with the cmex10 option to ensure that
% only type 1 fonts will utilized at all point sizes. Without this option,
% it is possible that some math symbols, particularly those within
% footnotes, will be rendered in bitmap form which will result in a
% document that can not be IEEE Xplore compliant!
%
% Also, note that the amsmath package sets \interdisplaylinepenalty to 10000
% thus preventing page breaks from occurring within multiline equations. Use:
%\interdisplaylinepenalty=2500
% after loading amsmath to restore such page breaks as IEEEtran.cls normally
% does. amsmath.sty is already installed on most LaTeX systems. The latest
% version and documentation can be obtained at:
% http://www.ctan.org/tex-archive/macros/latex/required/amslatex/math/





% *** SPECIALIZED LIST PACKAGES ***
%
%\usepackage{algorithmic}
% algorithmic.sty was written by Peter Williams and Rogerio Brito.
% This package provides an algorithmic environment fo describing algorithms.
% You can use the algorithmic environment in-text or within a figure
% environment to provide for a floating algorithm. Do NOT use the algorithm
% floating environment provided by algorithm.sty (by the same authors) or
% algorithm2e.sty (by Christophe Fiorio) as IEEE does not use dedicated
% algorithm float types and packages that provide these will not provide
% correct IEEE style captions. The latest version and documentation of
% algorithmic.sty can be obtained at:
% http://www.ctan.org/tex-archive/macros/latex/contrib/algorithms/
% There is also a support site at:
% http://algorithms.berlios.de/index.html
% Also of interest may be the (relatively newer and more customizable)
% algorithmicx.sty package by Szasz Janos:
% http://www.ctan.org/tex-archive/macros/latex/contrib/algorithmicx/




% *** ALIGNMENT PACKAGES ***
%
%\usepackage{array}
% Frank Mittelbach's and David Carlisle's array.sty patches and improves
% the standard LaTeX2e array and tabular environments to provide better
% appearance and additional user controls. As the default LaTeX2e table
% generation code is lacking to the point of almost being broken with
% respect to the quality of the end results, all users are strongly
% advised to use an enhanced (at the very least that provided by array.sty)
% set of table tools. array.sty is already installed on most systems. The
% latest version and documentation can be obtained at:
% http://www.ctan.org/tex-archive/macros/latex/required/tools/


%\usepackage{mdwmath}
%\usepackage{mdwtab}
% Also highly recommended is Mark Wooding's extremely powerful MDW tools,
% especially mdwmath.sty and mdwtab.sty which are used to format equations
% and tables, respectively. The MDWtools set is already installed on most
% LaTeX systems. The lastest version and documentation is available at:
% http://www.ctan.org/tex-archive/macros/latex/contrib/mdwtools/


% IEEEtran contains the IEEEeqnarray family of commands that can be used to
% generate multiline equations as well as matrices, tables, etc., of high
% quality.


%\usepackage{eqparbox}
% Also of notable interest is Scott Pakin's eqparbox package for creating
% (automatically sized) equal width boxes - aka "natural width parboxes".
% Available at:
% http://www.ctan.org/tex-archive/macros/latex/contrib/eqparbox/





% *** SUBFIGURE PACKAGES ***
%\usepackage[tight,footnotesize]{subfigure}
% subfigure.sty was written by Steven Douglas Cochran. This package makes it
% easy to put subfigures in your figures. e.g., "Figure 1a and 1b". For IEEE
% work, it is a good idea to load it with the tight package option to reduce
% the amount of white space around the subfigures. subfigure.sty is already
% installed on most LaTeX systems. The latest version and documentation can
% be obtained at:
% http://www.ctan.org/tex-archive/obsolete/macros/latex/contrib/subfigure/
% subfigure.sty has been superceeded by subfig.sty.



%\usepackage[caption=false]{caption}
%\usepackage[font=footnotesize]{subfig}
% subfig.sty, also written by Steven Douglas Cochran, is the modern
% replacement for subfigure.sty. However, subfig.sty requires and
% automatically loads Axel Sommerfeldt's caption.sty which will override
% IEEEtran.cls handling of captions and this will result in nonIEEE style
% figure/table captions. To prevent this problem, be sure and preload
% caption.sty with its "caption=false" package option. This is will preserve
% IEEEtran.cls handing of captions. Version 1.3 (2005/06/28) and later 
% (recommended due to many improvements over 1.2) of subfig.sty supports
% the caption=false option directly:
%\usepackage[caption=false,font=footnotesize]{subfig}
%
% The latest version and documentation can be obtained at:
% http://www.ctan.org/tex-archive/macros/latex/contrib/subfig/
% The latest version and documentation of caption.sty can be obtained at:
% http://www.ctan.org/tex-archive/macros/latex/contrib/caption/




% *** FLOAT PACKAGES ***
%
%\usepackage{fixltx2e}
% fixltx2e, the successor to the earlier fix2col.sty, was written by
% Frank Mittelbach and David Carlisle. This package corrects a few problems
% in the LaTeX2e kernel, the most notable of which is that in current
% LaTeX2e releases, the ordering of single and double column floats is not
% guaranteed to be preserved. Thus, an unpatched LaTeX2e can allow a
% single column figure to be placed prior to an earlier double column
% figure. The latest version and documentation can be found at:
% http://www.ctan.org/tex-archive/macros/latex/base/



%\usepackage{stfloats}
% stfloats.sty was written by Sigitas Tolusis. This package gives LaTeX2e
% the ability to do double column floats at the bottom of the page as well
% as the top. (e.g., "\begin{figure*}[!b]" is not normally possible in
% LaTeX2e). It also provides a command:
%\fnbelowfloat
% to enable the placement of footnotes below bottom floats (the standard
% LaTeX2e kernel puts them above bottom floats). This is an invasive package
% which rewrites many portions of the LaTeX2e float routines. It may not work
% with other packages that modify the LaTeX2e float routines. The latest
% version and documentation can be obtained at:
% http://www.ctan.org/tex-archive/macros/latex/contrib/sttools/
% Documentation is contained in the stfloats.sty comments as well as in the
% presfull.pdf file. Do not use the stfloats baselinefloat ability as IEEE
% does not allow \baselineskip to stretch. Authors submitting work to the
% IEEE should note that IEEE rarely uses double column equations and
% that authors should try to avoid such use. Do not be tempted to use the
% cuted.sty or midfloat.sty packages (also by Sigitas Tolusis) as IEEE does
% not format its papers in such ways.





% *** PDF, URL AND HYPERLINK PACKAGES ***
%
\usepackage{url}
% url.sty was written by Donald Arseneau. It provides better support for
% handling and breaking URLs. url.sty is already installed on most LaTeX
% systems. The latest version can be obtained at:
% http://www.ctan.org/tex-archive/macros/latex/contrib/misc/
% Read the url.sty source comments for usage information. Basically,
% \url{my_url_here}.

% *** Do not adjust lengths that control margins, column widths, etc. ***
% *** Do not use packages that alter fonts (such as pslatex).         ***
% There should be no need to do such things with IEEEtran.cls V1.6 and later.
% (Unless specifically asked to do so by the journal or conference you plan
% to submit to, of course. )


% correct bad hyphenation here
\hyphenation{op-tical net-works semi-conduc-tor}


\begin{document}
%
% paper title
% can use linebreaks \\ within to get better formatting as desired
\title{Securing and Integrating an Intranet of Things\\ within the Home}


% author names and affiliations
% use a multiple column layout for up to three different
% affiliations
\author{\IEEEauthorblockN{Fergus Leahy, Joseph Sventek, Paul Harvey}
\IEEEauthorblockA{School of Computing Science\\ University of Glasgow\\
Email: \{fergus.leahy, joseph.sventek\}@glasgow.ac.uk}
}

\maketitle


\begin{abstract}
The abstract goes here.
\end{abstract}

% no keywords

\section{Introduction}
The modern home is becoming increasingly filled with variety of connected devices, each providing a myriad of different and often overlapping services within the home. More recently, ``smart-appliances'' and the Internet of Things have began to enter our homes, attempting to digitize our already existing ``dumb-appliances'' and objects within the home. 

As these devices enter our homes in an often piecemeal fashion, bringing with them their own distinct ecosystems, protocols and standards, the user is faced with an increasingly difficult burden of managing this network of heterogeneous devices. Due to the sheer number and diversity of these devices, problems arise with respect to how these devices co-operate, as well as how to ensure the user's network and information remains secure against new and unanticipated threats.

Many existing devices present a severe lack of thoughtful integration into existing networks and infrastructure. A common approach has been to integrate traditional power-hungry 802.11 WIFI chipsets and offer services directly to Internet\cite{IETF_CORE,Xively} or connecting to a cloud service\cite{SmartThings,Twine}; whilst gaining the benefits of the Internet and Cloud (anywhere access, scalable power/storage), both also have multiple weaknesses by relying on such connectivity (reliability, security, privacy, data-ownership) and have resulted in attacks similar to Stuxnet where devices have come under attack\cite{IoTWorm}.

Talk about homework

This paper presents a solution to securing an Internet of Things, starting with securing an Intranet of Things within the home and ensuring devices within the home are safe against local threats. Secondly, this paper demonstrates integrating the IoT into a state-of-the-art smart router, Homework\cite{HomeworkProject}, for enabling a dynamic and powerful closed loop of control without the need for external connectivity to the Internet or Cloud. 

The main contributions of this paper are as follows:
\begin{itemize}
  \item A port of IoT protocol to TinyOS, previously implemented on the Contiki and Arduino platforms.
  \item Secure Internet of Things protocol, enabling new devices to securely join an existing network without a pre-shared key, providing authentication and end-to-end secrecy between Things within the network. 
  \item Homework Database Cache integration, a Java-based proxy enabling the IoT to communicate with a stream database, publish/subscribe engine and automata, to provide a closed loop of interaction.
\end{itemize}

\section{Background} % (fold)
\label{sec:background}

\subsection{Hardware and Software Platforms} % (fold)
\label{sec:hardware_and_software_platforms}

\subsection{Internet of Things Protocols} % (fold)
\label{sub:internet_of_things_protocols}
In our previous research\cite{KNoT}, existing protocols for traditional connectivity, domotics and IoTs were found to be inadequate for the typical IoT architecture and the types of devices it's designed to run on i.e. low power embedded devices. Existing protocols either cast a Thing to fit into the traditional RESTful client-server architecture\cite{IETF_COAP_HTTP}, are Cloud orientated\cite{SmartThings,Twine}, are extremely inefficient\cite{xAP} or are simply too heavyweight for such low power and constrained devices (TCP/IP), where battery power is a scarce resource. 

To remedy this, a model for the typical IoT was created consisting of sensors and actuators, as well as controllers which orchestrated the network. On top of this model a lightweight and efficient protocol was designed and implemented, ensuring robust, low power and scalable networking is possible with a network of Things ranging from 10s to 100s of devices.
% subsection internet_of_things_protocols (end)



% section hardware_and_software_platforms (end)
\subsection{Security in Wireless Sensor Networks} % (fold)
\label{sub:existing_security}
Deployable security in wireless sensor networks continues to be a significant problem for several reasons. Firstly, power is a major concern in WSNs, thus running expensive cryptography schemes in order to keep transmitted data secret can be detrimental to the lifetime of a node. Secondly, WSN nodes are also extremely constrained in terms of memory (ROM/RAM), requiring cryptography algorithms to fit within extreme size constraints and is especially problematic when trying to reduce computational load by storing pre-computed tables. Lastly, being able to dynamically add new nodes to a network post-deployment, as well as redistribute new keys for the network, enables the network to scale, replace failed nodes and protect against attackers. Previous work has demonstrated various solutions to parts of the problems, but have yet to create a complete solution and efficient solution for enabling nodes to dynamically join a secure network without some shared secret known a priori.

The majority of the previous work breaks down into two categories, symmetric and asymmetric cryptography. The rest of this section will discuss the most significant of these previous works of both types of cryptography.

TinySec\cite{TinySec}, a symmetric cryptography library for TinyOS 1.0, was some of the first work created to try and solve this security problem, intending to demonstrate that software-based cryptography was possible on constrained devices with a minimal power overhead, around 10\% for both encryption and authentication. To achieve this, TinySec was designed around WSNs extreme resource constraints, taking advantage of some of these constraints, such as the limited networking bandwidth, optimising the security primitives in order to reduce the security overhead added to each packet.

MiniSec, another symmetric cryptography library for TinyOS 1.0 and a successor to TinySec, was created to further improve the security provided by TinySec and add replay prevention whilst still maintaining the minimal overheads which TinySec achieved.
MiniSec paragraph \cite{MiniSec} mention failure to see a correlation between the paper and implementation provided (IV cheat and major flaws)

TinyPK paragraph \cite{TinyPK}

TinyECC paragraph \cite{TinyECC}

% subsection existing_security (end)
\subsection{Smart Home Router} % (fold)
\label{sub:integration}
Homework Cache \cite{InformationPlane} \cite{DEBSChallenge}
% subsection integration (end)

% section background (end)

\section{Aims} % (fold)
\label{sec:aims}

\subsection{Security Aims} % (fold)
\label{sub:security_aims}

Within security this paper aims to mitigate the following attack vectors:
\begin{itemize}
  \item Eavesdropping
  \item Masquerading
  \item Man-in-the-middle
\end{itemize}
As a result of protecting against the other forms of attack, the risk of node capture is substantially reduced.
% subsection security_aims (end)
\subsection{Integration Aims} % (fold)
\label{sub:integration_aims}

% subsection integration_aims (end)
% section design_goals  (end)
\section{Design} % (fold)
\label{sec:design}
\subsection{Security} % (fold)
\label{sub:security}
talk about combining asymmetric and symmetric to bootstrap keys and perform secure and authenticated handover
% subsection security (end)
\subsection{Integration} % (fold)
\label{sub:design_integration}
discuss architecture and use of automata to react to the network
% subsection design_integration (end)
% section design (end)

\section{Asymmetric Security} % (fold)
\label{sec:asymmetric_security}
Discuss the crypto used, the cert verification and key handover as well as optimisation
% section asymmetric_security (end)

\section{Symmetric Security} % (fold)
\label{sec:symmetric_security}
discuss the crypto used and MAC
as well as IV syncing and key refresh (skipjack)
% section symmetric_security (end)
\section{Integration} % (fold)
\label{sec:integration}
discuss impelemenation of automata and 
% section integration (end)

\section{Evaluation} % (fold)
\label{sec:evaluation}
Code size, ASym timings/optimisations.
% section evaluation (end)

\section{Conclusion}


\section{Future Work} % (fold)
\label{sec:future_work}

\begin{itemize}
  \item improve symmetric security (port AES)
  \item add internet connectivity securely
  \item implement on a wider range of devices
  \item port controller to PC to better support expensive PKC
\end{itemize}
% section future_work (end)

% conference papers do not normally have an appendix


% use section* for acknowledgement
\section*{Acknowledgment}


The authors would like to thank... many great peepz





% trigger a \newpage just before the given reference
% number - used to balance the columns on the last page
% adjust value as needed - may need to be readjusted if
% the document is modified later
%\IEEEtriggeratref{8}
% The "triggered" command can be changed if desired:
%\IEEEtriggercmd{\enlargethispage{-5in}}

% references section

% can use a bibliography generated by BibTeX as a .bbl file
% BibTeX documentation can be easily obtained at:
% http://www.ctan.org/tex-archive/biblio/bibtex/contrib/doc/
% The IEEEtran BibTeX style support page is at:
% http://www.michaelshell.org/tex/ieeetran/bibtex/
%\bibliographystyle{IEEEtran}
% argument is your BibTeX string definitions and bibliography database(s)
%\bibliography{IEEEabrv,../bib/paper}
%
% <OR> manually copy in the resultant .bbl file
% set second argument of \begin to the number of references
% (used to reserve space for the reference number labels box)
\bibliographystyle{./sty/IEEEtran}
\bibliography{./sty/examples/IEEEabrv,bibliography}




% that's all folks
\end{document}


% An example of a floating figure using the graphicx package.
% Note that \label must occur AFTER (or within) \caption.
% For figures, \caption should occur after the \includegraphics.
% Note that IEEEtran v1.7 and later has special internal code that
% is designed to preserve the operation of \label within \caption
% even when the captionsoff option is in effect. However, because
% of issues like this, it may be the safest practice to put all your
% \label just after \caption rather than within \caption{}.
%
% Reminder: the "draftcls" or "draftclsnofoot", not "draft", class
% option should be used if it is desired that the figures are to be
% displayed while in draft mode.
%
%\begin{figure}[!t]
%\centering
%\includegraphics[width=2.5in]{myfigure}
% where an .eps filename suffix will be assumed under latex, 
% and a .pdf suffix will be assumed for pdflatex; or what has been declared
% via \DeclareGraphicsExtensions.
%\caption{Simulation Results}
%\label{fig_sim}
%\end{figure}

% Note that IEEE typically puts floats only at the top, even when this
% results in a large percentage of a column being occupied by floats.


% An example of a double column floating figure using two subfigures.
% (The subfig.sty package must be loaded for this to work.)
% The subfigure \label commands are set within each subfloat command, the
% \label for the overall figure must come after \caption.
% \hfil must be used as a separator to get equal spacing.
% The subfigure.sty package works much the same way, except \subfigure is
% used instead of \subfloat.
%
%\begin{figure*}[!t]
%\centerline{\subfloat[Case I]\includegraphics[width=2.5in]{subfigcase1}%
%\label{fig_first_case}}
%\hfil
%\subfloat[Case II]{\includegraphics[width=2.5in]{subfigcase2}%
%\label{fig_second_case}}}
%\caption{Simulation results}
%\label{fig_sim}
%\end{figure*}
%
% Note that often IEEE papers with subfigures do not employ subfigure
% captions (using the optional argument to \subfloat), but instead will
% reference/describe all of them (a), (b), etc., within the main caption.


% An example of a floating table. Note that, for IEEE style tables, the 
% \caption command should come BEFORE the table. Table text will default to
% \footnotesize as IEEE normally uses this smaller font for tables.
% The \label must come after \caption as always.
%
%\begin{table}[!t]
%% increase table row spacing, adjust to taste
%\renewcommand{\arraystretch}{1.3}
% if using array.sty, it might be a good idea to tweak the value of
% \extrarowheight as needed to properly center the text within the cells
%\caption{An Example of a Table}
%\label{table_example}
%\centering
%% Some packages, such as MDW tools, offer better commands for making tables
%% than the plain LaTeX2e tabular which is used here.
%\begin{tabular}{|c||c|}
%\hline
%One & Two\\
%\hline
%Three & Four\\
%\hline
%\end{tabular}
%\end{table}


% Note that IEEE does not put floats in the very first column - or typically
% anywhere on the first page for that matter. Also, in-text middle ("here")
% positioning is not used. Most IEEE journals/conferences use top floats
% exclusively. Note that, LaTeX2e, unlike IEEE journals/conferences, places
% footnotes above bottom floats. This can be corrected via the \fnbelowfloat
% command of the stfloats package.

